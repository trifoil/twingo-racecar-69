\documentclass[a4paper, 12pt]{article}
\usepackage[french]{babel}
\usepackage[utf8]{inputenc}
\usepackage[T1]{fontenc}
\usepackage{graphicx}
\usepackage{geometry}
\usepackage{titlesec}
\usepackage{fancyhdr}
\usepackage{hyperref}
\usepackage{enumitem}
\usepackage{listings}
\usepackage{xcolor}
\usepackage{tikz} % For precise background positioning
\usepackage{eso-pic} % For adding content to every page

% Correction pour l'erreur de headheight
\setlength{\headheight}{14.5pt}

\setlist[itemize]{label=\textbullet}

% Configuration de la page
\geometry{left=2.5cm, right=2.5cm, top=2.5cm, bottom=2.5cm}

% Style des titres
\titleformat{\section}{\large\bfseries}{\thesection}{1em}{}
\titleformat{\subsection}{\normalsize\bfseries}{\thesubsection}{1em}{}

% Style des listings
\lstset{
	basicstyle=\ttfamily\small,
	breaklines=true,
	frame=single,
	backgroundcolor=\color{gray!10},
	keywordstyle=\color{blue},
	commentstyle=\color{green!50!black},
	stringstyle=\color{red},
	showstringspaces=false,
	literate=
	{é}{{\'e}}1
	{è}{{\`e}}1
	{ê}{{\^e}}1
	{ë}{{\"e}}1
	{É}{{\'E}}1
	{Ê}{{\^E}}1
	{à}{{\`a}}1
	{â}{{\^a}}1
	{ç}{{\c c}}1
	{Ç}{{\c C}}1
	{ù}{{\`u}}1
	{û}{{\^u}}1
}

% Add FWB logo to every page
\AddToShipoutPictureBG{%
	\begin{tikzpicture}[remember picture,overlay]
		\node[anchor=south east, xshift=-2.5cm, yshift=1cm] at (current page.south east)
		{\includegraphics[width=0.3\textwidth]{fwb.png}};
	\end{tikzpicture}%
	\begin{tikzpicture}[remember picture,overlay]
		\node[anchor=south west, xshift=2.5cm, yshift=0.75cm] at (current page.south west)
		{\includegraphics[width=0.2\textwidth]{logo.png}};
	\end{tikzpicture}%
	\begin{tikzpicture}[remember picture,overlay]
		\node[anchor=south west, xshift=1cm, yshift=1cm] at (current page.south west)
		{\includegraphics[width=0.02\textwidth]{side.png}};
	\end{tikzpicture}%
}

% En-tête et pied de page
\pagestyle{fancy}
\fancyhf{}
\rhead{\thepage}
\lhead{Projet Voiture - Groupe 5}
\renewcommand{\headrulewidth}{0.4pt}

% Page de garde
\title{}
\author{}
\date{}

% Custom title page with background image
\renewcommand{\maketitle}{%
	\begin{titlepage}
		% Background image
		\begin{tikzpicture}[remember picture,overlay]
			\node[anchor=north west, inner sep=0pt] at (current page.north west)
			{\includegraphics[width=\paperwidth]{entete.png}};
		\end{tikzpicture}
		
		% Content with proper vertical spacing
		\null  % Needed to start a new paragraph
		\vspace*{5cm} % Adjust this value to position your content lower
		
		\centering
		{\large Gestion de projet \\}
		\vspace{0.5cm}
		{\LARGE\textbf{Projet Voiture 2IRT} \\}
		\vspace{0.5cm}
		{\large Choix technologiques \\}
		
		\vspace{6cm}
		
		\begin{flushright} % Left alignment for the following content
			\textbf{Année Académique} \\
			2024 - 2025 \\
			\vspace{0.5cm}
			\textbf{Groupe} \\
			5 \\
			\vspace{0.5cm}
			\textbf{Membres} \\
			Colle Joulian \\
			Deneyer Tom \\
			Kruczynski Mathis \\
			Mauroit Antoine \\
			Staquet Esteban \\
			Vangeebergen Augustin \\
		\end{flushright}
	\end{titlepage}
}

\begin{document}
	% Page de garde
	\maketitle
	\thispagestyle{empty}
	\newpage
	
	% Table des matières
	\tableofcontents
	\thispagestyle{empty}
	\newpage
	
	% Corps du document
	\setcounter{page}{1}
	
	\section{Choix Technologiques Matériels}
	
	\subsection{Le Raspberry Pi comme Solution Optimale}
	
	Le choix du Raspberry Pi 3 Modèle B comme plateforme centrale pour ce projet s'est imposé face aux microcontrôleurs traditionnels comme Arduino pour plusieurs raisons fondamentales :
	
	\begin{itemize}
		\item \textbf{Puissance de calcul} : Contrairement aux microcontrôleurs classiques, le Raspberry Pi intègre un processeur quad-core à 1.2GHz et 1GB de RAM, permettant d'exécuter un système d'exploitation complet et de gérer simultanément :
		\begin{itemize}
			\item Le traitement des données des capteurs en temps réel
			\item Les algorithmes de navigation (relativement) complexes
			\item La communication réseau
		\end{itemize}
		
		\item \textbf{Connectivité intégrée} : Le Raspberry Pi offre nativement :
		\begin{itemize}
			\item Wi-Fi 802.11n et Bluetooth 4.1 pour les communications sans fil
			\item 4 ports USB pour connecter des périphériques additionnels
			\item Interface HDMI pour le débogage
		\end{itemize}
		
		\item \textbf{Multitasking} : La capacité à exécuter plusieurs processus en parallèle est cruciale pour :
		\begin{itemize}
			\item Gérer simultanément les capteurs et les moteurs
			\item Maintenir une connexion SSH active
			\item Exécuter un serveur web pour l'interface de contrôle
		\end{itemize}
		
		\item \textbf{Écosystème logiciel} : L'environnement Linux permet d'utiliser :
		\begin{itemize}
			\item Python 3 comme langage principal avec toutes ses bibliothèques
			\item Des outils professionnels de versioning (Git)
			\item Des frameworks de test et d'intégration continue
		\end{itemize}
	\end{itemize}
	
	\subsection{Limitations des Microcontrôleurs Traditionnels}
	
	Les solutions comme Arduino ou ESP32, bien que performantes pour certaines applications, présentent des limitations majeures pour ce projet :
	
	\begin{itemize}
		\item \textbf{Capacité de traitement insuffisante} pour les algorithmes avancés de détection d'obstacles et d'optimisation de trajectoire
		
		\item \textbf{Absence de système d'exploitation} rendant complexe :
		\begin{itemize}
			\item La gestion concurrente des tâches
			\item Le débogage avancé
			\item Les communications réseau
		\end{itemize}
		
		\item \textbf{Connectivité limitée} nécessitant des modules additionnels pour certaines fonctionnalités :
		\begin{itemize}
			\item Le stockage de données
			\item Les interfaces utilisateur avancées
		\end{itemize}
		
		\item \textbf{Espace mémoire restreint} incompatible avec :
		\begin{itemize}
			\item Le stockage des logs de diagnostic
			\item L'exécution de bibliothèques complexes
			\item La gestion de protocoles réseau complets
		\end{itemize}
	\end{itemize}
	
	\subsection{Comparaison avec d'Autres Solutions}
	
	\begin{table}[h]
		\centering
		\caption{Comparatif des plateformes matérielles}
		\begin{tabular}{|l|c|c|c|}
			\hline
			\textbf{Caractéristique} & \textbf{Raspberry Pi 3} & \textbf{Arduino Mega} & \textbf{ESP32} \\
			\hline
			Processeur & Quad-core 1.2GHz & 16MHz & Dual-core 240MHz \\
			\hline
			Mémoire & 1GB RAM & 8KB RAM & 520KB RAM \\
			\hline
			Système d'exploitation & Linux & Aucun & FreeRTOS \\
			\hline
			Connectivité & Wi-Fi/BT intégrés & Requiert shields & Wi-Fi/BT intégrés \\
			\hline
			Langages supportés & Python, C++, ... & C/C++ & C/C++, MicroPython \\
			\hline
			Prix & \textasciitilde50€ & \textasciitilde40€ & \textasciitilde5-10€ \\
			\hline
		\end{tabular}
	\end{table}
	
	Ce tableau montre clairement l'avantage du Raspberry Pi en termes de rapport performance/prix pour les besoins spécifiques de ce projet. Il est toutefois à noter que l'ESP32 présente un excellent compromis pour des applications plus légères avec sa connectivité sans fil intégrée.
	
	\subsection{Proposition d'alternative et son rejet}
	
	Bien que, comme mentionné précédemment, l’ESP32 présente certains avantages pour des applications légères, plusieurs facteurs critiques justifient le maintien des Raspberry Pi 3B dans ce projet pédagogique :
	
	\begin{itemize}
		\item \textbf{Compatibilité avec le parc existant} :
		\begin{itemize}
			\item L'école dispose déjà d'une flotte de Raspberry Pi 3B fonctionnels
			\item Tous les accessoires périphériques (écrans, câbles, alimentations) sont prévus pour le format Pi
			\item Les supports pédagogiques actuels sont spécifiquement conçus pour Raspberry Pi, sans compter la communauté et la documentation abondante disponible en ligne.
		\end{itemize}
		
		\item \textbf{Avantages pédagogiques} :
		\begin{itemize}
			\item Initiation à Linux, système d'exploitation professionnel
			\item Environnement de développement complet (IDE, debuggeur)
			\item Meilleure visualisation des concepts avec interface graphique
			\item Compatibilité avec les outils standards (Git, SSH, VNC)
		\end{itemize}
		
		\item \textbf{Limitations techniques de l'ESP32} :
		\begin{itemize}
			\item Impossible d'exécuter simultanément :
			\begin{itemize}
				\item L'interface utilisateur
				\item Le serveur web de contrôle
				\item Le logging avancé des données
			\end{itemize}
			\item Mémoire insuffisante pour le débogage interactif
			\item Connectivité HDMI absente nécessitant des adaptateurs supplémentaires
		\end{itemize}
		
		\item \textbf{Économie à court terme, coût à long terme} :
		\begin{itemize}
			\item Coût de migration des développements existants
			\item Nécessité de reformer les enseignants et étudiants
			\item Achat de nouveaux accessoires (programmateurs, shields)
			\item Perte de compatibilité avec les projets antérieurs
		\end{itemize}
		
		\newpage
		
		\item \textbf{Benchmarks comparatifs} : Les tests pratiques démontrent des limitations majeures de l'ESP32 pour notre cas d'usage :
		
		
		
		\begin{table}[h]
			\centering
			\caption{Comparaison des performances réelles}
			\begin{tabular}{|l|c|c|}
				\hline
				\textbf{Tâche} & \textbf{RPi 3B} & \textbf{ESP32} \\
				\hline
				Traitement d'image (640x480) & 15 fps & Impossible \\
				\hline
				Serveur web + interface & Fluide & Non réalisable \\
				\hline
				Stockage des logs (1h) & Direct & Nécessite SD card \\
				\hline
				Débogage temps réel & Possible & Très limité \\
				\hline
			\end{tabular}
		\end{table}
	\end{itemize}
	
	\textbf{Conclusion} : Bien que techniquement envisageable pour des applications minimalistes, le remplacement des Raspberry Pi par des ESP32 compromettrait la qualité pédagogique et l'expérience d'apprentissage, sans réelle économie à l'échelle du parc existant. Et c'est sans compter l'affichage d'images de la webcam dans le cas où cette fonctionnalité serait implémentée dans les temps.
	
	
	\section{Choix Technologiques Logiciels}
	
	\subsection{Système d'Exploitation}
	
	Le choix s'est porté sur \textbf{Raspberry Pi OS} (version GUI, 32bits) pour plusieurs raisons techniques :
	
	\begin{itemize}
		\item \textbf{Compatibilité matérielle} : Support natif de tous les composants du projet
		\item \textbf{Stabilité} : Distribution officielle maintenue par la fondation Raspberry Pi (mieux qu'une distribution obscure IOT pour laquelle certains drivers manqueraient).
		\item \textbf{Outils intégrés} : Inclut par défaut :
		\begin{itemize}
			\item Gestionnaire de paquets apt
			\item Supports Hotspot et WiFi 
		\end{itemize}
	\end{itemize}
	
	
	\subsection{Langage de Programmation}
	
	Python 3 a été sélectionné comme langage principal pour :
	
	\begin{itemize}
		\item \textbf{Productivité} : Syntaxe claire permettant un développement rapide
		\item \textbf{Polyvalence des paradigmes} :
		\begin{itemize}
			\item Prise en charge quasi-complète de la programmation orientée objet (classes, héritage, polymorphisme)
			\item Possibilité de programmer de manière procédurale pour les scripts simples
			\item Éléments de programmation fonctionnelle (fonctions lambda, map, filter)
			\item Flexibilité permettant d'adapter le style de programmation à chaque composant du système
		\end{itemize}
		
		\item \textbf{Bibliothèques spécialisées} :
		\begin{itemize}
			\item board (récupération des lignes sda/scl par défaut)
			\item busio (pour l'utilisation des bus I2C)
			\item adafruit\_ina219
			\item adafruit\_pca9685
			\item adafruit\_tcs34725
			\item rpi.gpio
			\item unittest
		\end{itemize}
		
		\item \textbf{Communauté active} : Large base d'utilisateurs et documentation abondante
		\item \textbf{Portabilité} : Fonctionne sur tous les systèmes d'exploitation (ou presque)
	\end{itemize}
	
	
	
	\subsection{Outils de Développement}
	
	L'environnement logiciel comprend :
	
	\begin{itemize}
		\item \textbf{Git - Github} : Pour le versioning et la collaboration en équipe
		\item \textbf{SSH/TightVNC} : Permet le développement à distance
		\item \textbf{UnitTest} : Framework de tests unitaires intégré à Python
			\end{itemize}
	L'environnement non-logiciel comprend :
		\begin{itemize}
		\item \textbf{Notion} : 
				\begin{itemize}
					\item pour la gestion des taches en temps réel pour le scrumboard
					\item visualisation du planning
					\item centraliser la documentation
					\item gérer les daily scrums
					\item reviews
					\item retrospectives
					\item journal de bord accessible par toute l'équipe
					\item  gestion des user-stories
				 
				\end{itemize} 
			\end{itemize}
	
	\subsection{Architecture Logicielle}
	
	L'application suit une architecture modulaire :
	
	\begin{itemize}
		\item \textbf{Couche matérielle} : Drivers pour les capteurs et actionneurs
		\item \textbf{Couche contrôle} : Algorithmes de navigation et détection, tests unitaires
		\item \textbf{Couche communication} : Gestion des connexions réseau, SSH, I2C
		\item \textbf{Interface utilisateur} : TUI à travers le SSH
	\end{itemize}
	Cette architecture permet une maintenance aisée et une bonne séparation des préoccupations.
	
	\subsection{Approche de Test et Validation}
	
	Une méthodologie rigoureuse de test a été mise en place pour garantir la fiabilité du système tout en protégeant le matériel :
	
	\begin{itemize}
		\item \textbf{Philosophie de test} :
		\begin{itemize}
			\item \textbf{Tests avant implémentation} : Développement piloté par les tests (TDD) pour les composants critiques
			\item \textbf{Protection du matériel} : Validation logicielle complète avant tout déploiement sur le Raspberry Pi, utilisation de capteur de protection et de préactionneurs (pont en H)
			\item \textbf{Isolation des défauts} : Tests unitaires pour identifier précisément l'origine des problèmes
		\end{itemize}
		
		\item \textbf{Infrastructure de test} :
		\begin{itemize}
			\item Framework \texttt{unittest} de Python comme base
			\item Création de \textbf{mock objects} pour :
			\begin{itemize}
				\item Simuler les capteurs (GPIO, I2C) sans risque de court-circuit
				\item Émuler des conditions extrêmes (tension trop élevée, température critique)
				\item Tester les cas d'erreur sans endommager le matériel
			\end{itemize}
			\item Bancs de test virtuels pour valider les algorithmes de contrôle
		\end{itemize}
		
		\item \textbf{Exemple concret de mock} :
		\begin{lstlisting}[language=Python]
import unittest
from unittest.mock import MagicMock, patch
from vehicle import VehicleController

class MockMotor():
	def __init__(self, name):
		self.status = f"Motor {name} initialized"

	def activate(self):
		print("Motor activation signal sent")
		return "Motor activated"

class TestVehicleSystem(unittest.TestCase):

	@patch("vehicle.Motor", new=MockMotor)
	def test_vehicle_control(self):
		controller = VehicleController("main_motor", 100)
		result = controller.activate_motor()
		self.assertEqual(result, "Motor activated")
		print("Test passed: Motor control OK")

if __name__ == "__main__":
	unittest.main()
		\end{lstlisting}
		
		\item \textbf{Avantages de cette approche} :
		\begin{itemize}
			\item \textbf{Sécurité matérielle} : Aucun risque de griller des composants pendant le développement
			\item \textbf{Reproductibilité} : Conditions de test parfaitement contrôlées
			\item \textbf{Efficacité} : Tests exécutables sans le matériel physique
			\item \textbf{Couverture} : Possibilité de simuler des cas rares ou dangereux
		\end{itemize}
		
		\item \textbf{Validation progressive} :
		\begin{enumerate}
			\item Tests unitaires avec mocks (100\% de couverture)
			\item Tests d'intégration en environnement simulé
			\item Tests système sur banc de validation
			\item Déploiement sur le prototype physique
		\end{enumerate}
	\end{itemize}
	
	Cette méthodologie nous permet de développer en toute sécurité des fonctionnalités complexes comme le contrôle PID des moteurs ou la gestion des capteurs I2C, tout en minimisant les risques pour le matériel coûteux. Les mocks reproduisent fidèlement le comportement des composants physiques, y compris leurs temps de réponse et modes de défaillance.
	
\end{document}