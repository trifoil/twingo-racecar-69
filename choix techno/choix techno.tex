\documentclass[a4paper, 12pt]{article}
\usepackage[french]{babel}
\usepackage[utf8]{inputenc}
\usepackage[T1]{fontenc}
\usepackage{graphicx}
\usepackage{geometry}
\usepackage{titlesec}
\usepackage{fancyhdr}
\usepackage{hyperref}
\usepackage{enumitem}
\usepackage{listings}
\usepackage{xcolor}
\usepackage{tikz} % For precise background positioning
\usepackage{eso-pic} % For adding content to every page

% Correction pour l'erreur de headheight
\setlength{\headheight}{14.5pt}

\setlist[itemize]{label=\textbullet}

% Configuration de la page
\geometry{left=2.5cm, right=2.5cm, top=2.5cm, bottom=2.5cm}

% Style des titres
\titleformat{\section}{\large\bfseries}{\thesection}{1em}{}
\titleformat{\subsection}{\normalsize\bfseries}{\thesubsection}{1em}{}

% Style des listings
\lstset{
	basicstyle=\ttfamily\small,
	breaklines=true,
	frame=single,
	backgroundcolor=\color{gray!10},
	keywordstyle=\color{blue},
	commentstyle=\color{green!50!black},
	stringstyle=\color{red},
	showstringspaces=false,
	literate=
	{é}{{\'e}}1
	{è}{{\`e}}1
	{ê}{{\^e}}1
	{ë}{{\"e}}1
	{É}{{\'E}}1
	{Ê}{{\^E}}1
	{à}{{\`a}}1
	{â}{{\^a}}1
	{ç}{{\c c}}1
	{Ç}{{\c C}}1
	{ù}{{\`u}}1
	{û}{{\^u}}1
}

% Add FWB logo to every page
\AddToShipoutPictureBG{%
	\begin{tikzpicture}[remember picture,overlay]
		\node[anchor=south east, xshift=-2.5cm, yshift=1cm] at (current page.south east)
		{\includegraphics[width=0.3\textwidth]{fwb.png}};
	\end{tikzpicture}%
	\begin{tikzpicture}[remember picture,overlay]
		\node[anchor=south west, xshift=2.5cm, yshift=0.75cm] at (current page.south west)
		{\includegraphics[width=0.2\textwidth]{logo.png}};
	\end{tikzpicture}%
	\begin{tikzpicture}[remember picture,overlay]
		\node[anchor=south west, xshift=1cm, yshift=1cm] at (current page.south west)
		{\includegraphics[width=0.02\textwidth]{side.png}};
	\end{tikzpicture}%
}

% En-tête et pied de page
\pagestyle{fancy}
\fancyhf{}
\rhead{\thepage}
\lhead{Projet Voiture - Groupe X}
\renewcommand{\headrulewidth}{0.4pt}

% Page de garde
\title{}
\author{}
\date{}

% Custom title page with background image
\renewcommand{\maketitle}{%
	\begin{titlepage}
		% Background image
		\begin{tikzpicture}[remember picture,overlay]
			\node[anchor=north west, inner sep=0pt] at (current page.north west)
			{\includegraphics[width=\paperwidth]{entete.png}};
		\end{tikzpicture}
		
		% Content with proper vertical spacing
		\null  % Needed to start a new paragraph
		\vspace*{5cm} % Adjust this value to position your content lower
		
		\centering
		{\large Gestion de projet \\}
		\vspace{0.5cm}
		{\LARGE\textbf{Projet Voiture 2IRT} \\}
		\vspace{0.5cm}
		{\large Choix technologiques \\}
		
		\vspace{6cm}
		
		\begin{flushright} % Left alignment for the following content
			\textbf{Année Académique} \\
			2024 - 2025 \\
			\vspace{0.5cm}
			\textbf{Groupe} \\
			5 \\
			\vspace{0.5cm}
			\textbf{Membres} \\
			Colle Joulian \\
			Deneyer Tom \\
			Kruczynski Mathis \\
			Mauroit Antoine \\
			Staquet Esteban \\
			Vangeebergen Augustin \\
		\end{flushright}
	\end{titlepage}
}

\begin{document}
	% Page de garde
	\maketitle
	\thispagestyle{empty}
	\newpage
	
	% Table des matières
	\tableofcontents
	\thispagestyle{empty}
	\newpage
	
	% Corps du document
	\setcounter{page}{1}
	
	\section{Choix Technologiques Matériels}
	
	\subsection{Le Raspberry Pi comme Solution Optimale}
	
	Le choix du Raspberry Pi 3 Modèle B comme plateforme centrale pour ce projet s'est imposé face aux microcontrôleurs traditionnels comme Arduino pour plusieurs raisons fondamentales :
	
	\begin{itemize}
		\item \textbf{Puissance de calcul} : Contrairement aux microcontrôleurs classiques, le Raspberry Pi intègre un processeur quad-core à 1.2GHz et 1GB de RAM, permettant d'exécuter un système d'exploitation complet (Raspbian) et de gérer simultanément :
		\begin{itemize}
			\item Le traitement des données des capteurs en temps réel
			\item Les algorithmes de navigation (relativement) complexes
			\item La communication réseau
		\end{itemize}
		
		\item \textbf{Connectivité intégrée} : Le Raspberry Pi offre nativement :
		\begin{itemize}
			\item Wi-Fi 802.11n et Bluetooth 4.1 pour les communications sans fil
			\item 4 ports USB pour connecter des périphériques additionnels
			\item Interface HDMI pour le débogage
		\end{itemize}
		
		\item \textbf{Multitâching} : La capacité à exécuter plusieurs processus en parallèle est cruciale pour :
		\begin{itemize}
			\item Gérer simultanément les capteurs et les moteurs
			\item Maintenir une connexion SSH active
			\item Exécuter un serveur web pour l'interface de contrôle
		\end{itemize}
		
		\item \textbf{Écosystème logiciel} : L'environnement Linux permet d'utiliser :
		\begin{itemize}
			\item Python 3 comme langage principal avec toutes ses bibliothèques
			\item Des outils professionnels de versioning (Git)
			\item Des frameworks de test et d'intégration continue
		\end{itemize}
	\end{itemize}
	
	\subsection{Limitations des Microcontrôleurs Traditionnels}
	
	Les solutions comme Arduino ou STM32, bien que performantes pour certaines applications, présentent des limitations majeures pour ce projet :
	
	\begin{itemize}
		\item \textbf{Capacité de traitement insuffisante} pour les algorithmes avancés de détection d'obstacles et d'optimisation de trajectoire
		
		\item \textbf{Absence de système d'exploitation} rendant complexe :
		\begin{itemize}
			\item La gestion concurrente des tâches
			\item Le débogage avancé
			\item Les communications réseau
		\end{itemize}
		
		\item \textbf{Connectivité limitée} nécessitant des shields additionnels pour :
		\begin{itemize}
			\item Le Wi-Fi/Bluetooth
			\item Le stockage de données
			\item Les interfaces utilisateur
		\end{itemize}
		
		\item \textbf{Espace mémoire restreint} incompatible avec :
		\begin{itemize}
			\item Le stockage des logs de diagnostic
			\item L'exécution de bibliothèques complexes
			\item La gestion de protocoles réseau complets
		\end{itemize}
	\end{itemize}
	
	\subsection{Comparaison avec d'Autres Solutions}
	
	\begin{table}[h]
		\centering
		\caption{Comparatif des plateformes matérielles}
		\begin{tabular}{|l|c|c|c|}
			\hline
			\textbf{Caractéristique} & \textbf{Raspberry Pi 3} & \textbf{Arduino Mega} & \textbf{STM32F4} \\
			\hline
			Processeur & Quad-core 1.2GHz & 16MHz & 180MHz \\
			\hline
			Mémoire & 1GB RAM & 8KB RAM & 192KB RAM \\
			\hline
			Système d'exploitation & Linux & Aucun & RTOS optionnel \\
			\hline
			Connectivité & Wi-Fi/BT intégrés & Requiert shields & Requiert modules \\
			\hline
			Langages supportés & Python, C++, etc. & C/C++ & C/C++ \\
			\hline
			Prix & \textasciitilde35€ & \textasciitilde40€ & \textasciitilde25€ \\
			\hline
		\end{tabular}
	\end{table}
	
	Ce tableau montre clairement l'avantage du Raspberry Pi en termes de rapport performance/prix pour les besoins spécifiques de ce projet.
	
	\section{Choix Technologiques Logiciels}
	
	\subsection{Système d'Exploitation}
	
	Le choix s'est porté sur \textbf{Raspbian} (version GUI) pour plusieurs raisons techniques :
	
	\begin{itemize}
		\item \textbf{Compatibilité matérielle} : Support natif de tous les composants du projet
		\item \textbf{Stabilité} : Distribution officielle maintenue par la fondation Raspberry Pi
		\item \textbf{Outils intégrés} : Inclut par défaut :
		\begin{itemize}
			\item Gestionnaire de paquets apt
			\item Interfaces de configuration matérielle
			\item Environnement de développement complet
		\end{itemize}
	\end{itemize}
	
	\subsection{Langage de Programmation}
	
	Python 3 a été sélectionné comme langage principal pour :
	
	\begin{itemize}
		\item \textbf{Productivité} : Syntaxe claire permettant un développement rapide
		\item \textbf{Bibliothèques spécialisées} :
		\begin{itemize}
			\item RPi.GPIO pour le contrôle des GPIO
			\item NumPy pour les calculs matriciels
			\item Matplotlib pour la visualisation des données
		\end{itemize}
		
		\item \textbf{Communauté active} : Large base d'utilisateurs et documentation abondante
		\item \textbf{Portabilité} : Fonctionne sur tous les systèmes d'exploitation
	\end{itemize}
	
	\subsection{Outils de Développement}
	
	L'environnement logiciel comprend :
	
	\begin{itemize}
		\item \textbf{Git} : Pour le versioning et la collaboration en équipe
		\item \textbf{SSH/TightVNC} : Permet le développement à distance
		\item \textbf{UnitTest} : Framework de tests unitaires intégré à Python
	\end{itemize}
	
	\subsection{Architecture Logicielle}
	
	L'application suit une architecture modulaire :
	
	\begin{itemize}
		\item \textbf{Couche matérielle} : Drivers pour les capteurs et actionneurs
		\item \textbf{Couche contrôle} : Algorithmes de navigation et détection
		\item \textbf{Couche communication} : Gestion des connexions réseau
		\item \textbf{Interface utilisateur} : Console et interface web
	\end{itemize}
	
	Cette architecture permet une maintenance aisée et une bonne séparation des préoccupations.
\end{document}